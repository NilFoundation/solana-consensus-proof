\subsection{Redshift Verification}
\textbf{WIP}


Redshift circuit repeats all steps from Section \ref{section:protocol:verifier}.
The verification circuit is a part of bridge design, and it is supposed that any output of the basic proof is an input to the verification circuit. 
Thus, we do not suppose any decoding for the proof because it can be represented directly in the desirable form. 

In the previous sections, we described circuits for most of the steps of the verifier algorithm.
However, steps \ref{algoeirhm:verifier:15}-\ref{algoeirhm:verifier:16} require additional clarification.

We consider step \ref{algoeirhm:verifier:16} firstly as a simpler one.
It contains basic arithmetic operations over finite field elements. 
These operations can be done with standard generic PLONK gate:
\begin{center}
	$\textbf{q}_L \cdot w_0 + \textbf{q}_R \cdot w_1 + \textbf{q}_M \cdot w_0 \cdot w_1 + \textbf{q}_O \cdot w_2 + \textbf{q}_C$
\end{center}
There are more optimal ways to perform these calculations.
However, the number of arithmetic operations is much less than in  Step \ref{algoeirhm:verifier:15}.
It means that any optimizations do not decrease prover or verifier complexities in any noticeable way. 

\paragraph{FRI Verification} is the main part of Step \ref{algoeirhm:verifier:15}. 
It contains two operations: Merkle tree path check and polynomial interpolation. 
The circuit version of Merkle path check algorithm does not differ from the original one. 
The circuit form Section \ref{section:poseidon} is used to check hash operations correctness. 

To check polynomial interpolation, the following circuit is used:
\begin{center}
    \begin{tabular}{ c|c|c|c|c|c|c|c|c|c }
        & $w_0$  & $w_1$  & $w_2$  & $w_3$  & $w_4$  & $w_5$ & $w_6$ & $w_7$ & $w_8$  \\
        \hline
        j + 0  & $a_0$  & $a_1$ & $s_0$ & $s_1$ & $x$ & $y$ & $\alpha$ & $\beta$ & $\dots$   \\
    \end{tabular}
\end{center}

Constraints ($\textbf{max degree} = 2$):
\begin{enumerate}
    \item $w_6 \cdot w_0 + w_7 = w_2 \longleftrightarrow 
            \alpha \cdot a_0 + \beta = s_0$
    \item $w_6 \cdot w_0 + w_7 = w_2 \longleftrightarrow 
            \alpha \cdot a_1 + \beta = s_1$
    \item $w_6 \cdot w_0 + w_7 = w_2 \longleftrightarrow 
            \alpha \cdot x + \beta = y$
\end{enumerate}

Copy constraints:
\begin{enumerate}
    \item $a_0, a_1, s_0, s_1, y$ are constrained by public input.
\end{enumerate}

The gate uses the line equation to check that all three points are on the same line.
This means, it checks $f(a_0) = s_0$ , $f(a_1) = s_1$, $f(x) = y$ for $f(X) = \alpha \cdot X + \beta$.
